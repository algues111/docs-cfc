%% Generated by Sphinx.
\def\sphinxdocclass{report}
\documentclass[letterpaper,10pt,french]{sphinxmanual}
\ifdefined\pdfpxdimen
   \let\sphinxpxdimen\pdfpxdimen\else\newdimen\sphinxpxdimen
\fi \sphinxpxdimen=.75bp\relax
\ifdefined\pdfimageresolution
    \pdfimageresolution= \numexpr \dimexpr1in\relax/\sphinxpxdimen\relax
\fi
%% let collapsible pdf bookmarks panel have high depth per default
\PassOptionsToPackage{bookmarksdepth=5}{hyperref}

\PassOptionsToPackage{booktabs}{sphinx}
\PassOptionsToPackage{colorrows}{sphinx}

\PassOptionsToPackage{warn}{textcomp}
\usepackage[utf8]{inputenc}
\ifdefined\DeclareUnicodeCharacter
% support both utf8 and utf8x syntaxes
  \ifdefined\DeclareUnicodeCharacterAsOptional
    \def\sphinxDUC#1{\DeclareUnicodeCharacter{"#1}}
  \else
    \let\sphinxDUC\DeclareUnicodeCharacter
  \fi
  \sphinxDUC{00A0}{\nobreakspace}
  \sphinxDUC{2500}{\sphinxunichar{2500}}
  \sphinxDUC{2502}{\sphinxunichar{2502}}
  \sphinxDUC{2514}{\sphinxunichar{2514}}
  \sphinxDUC{251C}{\sphinxunichar{251C}}
  \sphinxDUC{2572}{\textbackslash}
\fi
\usepackage{cmap}
\usepackage[T1]{fontenc}
\usepackage{amsmath,amssymb,amstext}
\usepackage{babel}



\usepackage{tgtermes}
\usepackage{tgheros}
\renewcommand{\ttdefault}{txtt}



\usepackage[Sonny]{fncychap}
\ChNameVar{\Large\normalfont\sffamily}
\ChTitleVar{\Large\normalfont\sffamily}
\usepackage{sphinx}

\fvset{fontsize=auto}
\usepackage{geometry}


% Include hyperref last.
\usepackage{hyperref}
% Fix anchor placement for figures with captions.
\usepackage{hypcap}% it must be loaded after hyperref.
% Set up styles of URL: it should be placed after hyperref.
\urlstyle{same}

\addto\captionsfrench{\renewcommand{\contentsname}{Contents:}}

\usepackage{sphinxmessages}
\setcounter{tocdepth}{1}



\title{Modules CFC IBM}
\date{mars 27, 2024}
\release{0.1}
\author{Arthur LE GUENNEC}
\newcommand{\sphinxlogo}{\vbox{}}
\renewcommand{\releasename}{Version}
\makeindex
\begin{document}

\ifdefined\shorthandoff
  \ifnum\catcode`\=\string=\active\shorthandoff{=}\fi
  \ifnum\catcode`\"=\active\shorthandoff{"}\fi
\fi

\pagestyle{empty}
\sphinxmaketitle
\pagestyle{plain}
\sphinxtableofcontents
\pagestyle{normal}
\phantomsection\label{\detokenize{index::doc}}


\sphinxstepscope


\chapter{Module 362 : Mettre en service les systèmes vocaux et vidéo complexes}
\label{\detokenize{Documentation-M362:module-362-mettre-en-service-les-systemes-vocaux-et-video-complexes}}\label{\detokenize{Documentation-M362::doc}}
\noindent{\hspace*{\fill}\sphinxincludegraphics[height=32\sphinxpxdimen]{{3CX_Logo_-_Wiki}.png}}


\section{3CX}
\label{\detokenize{Documentation-M362:cx}}

\subsection{Qu’est\sphinxhyphen{}ce que 3CX ?}
\label{\detokenize{Documentation-M362:qu-est-ce-que-3cx}}
\sphinxAtStartPar
3CX est une \sphinxstylestrong{solution de communications virtuelle} qui permet aux entreprises de gérer leurs appels téléphoniques, leur messagerie instantanée, leur vidéoconférence ainsi que tous les services que pourrait proposer un PBX classique, grâce à différentes installations et forfaits.

\sphinxAtStartPar
12 millions d’utilisateurs l’utilisent chaque jour, le placant donc sans souci sur le podium des leaders mondiaux de la téléphonie !

\noindent{\hspace*{\fill}\sphinxincludegraphics{{3cx-paccueil}.png}\hspace*{\fill}}


\bigskip\hrule\bigskip



\subsection{Licences}
\label{\detokenize{Documentation-M362:licences}}
\sphinxAtStartPar
Pour avoir une vue d’ensemble plus concrète de ce que propose 3CX en tant que service ou système, il est important de connaître les différentes licences proposées par l’entreprise du même nom.

\noindent{\hspace*{\fill}\sphinxincludegraphics{{3cx-licences}.png}\hspace*{\fill}}


\bigskip\hrule\bigskip



\subsection{Différents types d’installation}
\label{\detokenize{Documentation-M362:differents-types-d-installation}}
\sphinxAtStartPar


\sphinxAtStartPar



\subsubsection{Linux}
\label{\detokenize{Documentation-M362:linux}}
\sphinxAtStartPar


\sphinxAtStartPar
Chez 3CX, le principal avantage de leur système est leur \sphinxstylestrong{OS basé sur Debian} qui est \sphinxstylestrong{optimisé pour leur PBX virtuel.}

\sphinxAtStartPar
L’installation s’avère simple si l’on est familiarisé avec le milieu de la virtualisation sous linux.

\sphinxAtStartPar


\sphinxAtStartPar
\sphinxstylestrong{Votre machine a besoin d’au moins 1 cœur de processeur dédié et de 2Go de RAM. Si vous hébergez vous\sphinxhyphen{}même votre machine et que votre hébergeur utilise une unité centrale partagée, vous avez besoin de 2 cœurs !}

\sphinxAtStartPar
Passez en revue les spécifications matérielles suggérées afin d’allouer du temps d’unité centrale et des ressources de mémoire vive supplémentaires en fonction des critères suivants :
\begin{itemize}
\item {} 
\sphinxAtStartPar
Nombre d’appels simultanés gérés par le système.

\item {} 
\sphinxAtStartPar
Nombre d’utilisateurs actifs \sphinxhyphen{} 100 sessions actives du client Web sont plus exigeantes que 100 appels occasionnels via des téléphones IP.

\item {} 
\sphinxAtStartPar
L’utilisation de l’enregistrement des appels \sphinxhyphen{} le système est sollicité pour le mixage audio et le stockage des fichiers.

\end{itemize}
\begin{quote}

\sphinxAtStartPar
3CX peut être installé sur n’importe quel système fonctionnant sous Debian 12. Si vous souhaitez faire une installation barebone, assurez\sphinxhyphen{}vous que le matériel fonctionne avec Debian 12 et que le fournisseur du matériel vous aidera en cas de problème. Nous ne pouvons pas vous aider à résoudre les problèmes liés à l’installation de Debian 10 sur du matériel «  barebone « .

\sphinxAtStartPar
Ne configurez pas de réseau virtuel, d’interface VPN ou d’option TeamViewer VPN sur l’hôte 3CX.
\end{quote}

\sphinxAtStartPar
Pour les autres prérequis liés à la virtualisation, au réseau etc… Je vous invite à vous référer à leur documentation complète disponible avec le lien ci\sphinxhyphen{}dessous :

\sphinxAtStartPar
\sphinxstylestrong{https://www.3cx.fr/docs/manuel/installation\sphinxhyphen{}debian\sphinxhyphen{}linux\sphinxhyphen{}ipbx/}

\sphinxAtStartPar
Voici aussi le lien pour le téléchargement de l’iso linux de 3CX :

\sphinxAtStartPar
\sphinxstylestrong{https://downloads\sphinxhyphen{}global.3cx.com/downloads/debian10iso/debian\sphinxhyphen{}amd64\sphinxhyphen{}netinst\sphinxhyphen{}3cx.iso}

\sphinxAtStartPar


\sphinxAtStartPar
\sphinxstyleemphasis{Premier lancement de l’iso de 3CX}

\sphinxAtStartPar
Choisissez ce que vous préférez en fonction de vos habitudes d’installation de distributions Linux.

\sphinxAtStartPar


\sphinxAtStartPar
\sphinxstyleemphasis{Attendre que l’installation s’effectue et choisir les options correspondantes à vos besoins (FQDN…)}


\bigskip\hrule\bigskip


\sphinxAtStartPar


\sphinxAtStartPar


\sphinxAtStartPar
Lorsque votre VM aura redémarré et que vous aurez cette interface de disponible, je vous conseille d’installer 3CX avec votre navigateur web comme support visuel.

\sphinxAtStartPar

\begin{quote}

\sphinxAtStartPar
L’installation en CLI étant réservée aux utilisateurs aguerris de 3CX, je ne le vous recommanderais seulement si vous nécessitez de paramètres spéciaux/avancés.
\end{quote}

\sphinxAtStartPar


\noindent\sphinxincludegraphics{{conf-3cx1}.png}

\sphinxAtStartPar

\begin{itemize}
\item {} 
\sphinxAtStartPar
\sphinxstyleemphasis{Upload a new configuration file create on 3CX}

\item {} 
\sphinxAtStartPar
\sphinxstyleemphasis{Restore a backup}

\item {} 
\sphinxAtStartPar
\sphinxstyleemphasis{Install without config file (legacy, not recommended)}

\end{itemize}

\sphinxAtStartPar
Nous utiliserons la 3ème option pour cette installation.

\sphinxAtStartPar


\noindent\sphinxincludegraphics{{conf-3cx2}.png}

\sphinxAtStartPar


\sphinxAtStartPar
Cette étape nous permet de configurer les différents ports utilisés par les services de 3CX.
\begin{quote}

\sphinxAtStartPar
Si seulement votre instance 3CX tourne sur votre VM, je vous conseille de laiser les ports par défaut proposer par le wizard d’installation.

\sphinxAtStartPar
Dans le cas contraire, utilisez des ports qui ne sont pas utilisés par d’autres services!
\end{quote}


\bigskip\hrule\bigskip


\noindent{\hspace*{\fill}\sphinxincludegraphics[height=32\sphinxpxdimen]{{Windows_Logo_2012-2015}.png}}


\subsubsection{Windows}
\label{\detokenize{Documentation-M362:windows}}
\sphinxAtStartPar
Il est aussi possible d’héberger votre PBX 3CX sous l’OS Windows.

\sphinxAtStartPar
{\color{red}\bfseries{}|:warning:|} \sphinxstylestrong{DISCLAIMER} {\color{red}\bfseries{}|:warning:|}

\sphinxAtStartPar
Cependant, cela nécessitera des connaissances avancées, car vous vous retrouverez face à des contraintes plus récurrentes que sur Linux.

\sphinxAtStartPar
Par exemple, lors des MàJ Windows, il est possible que l’état du Firewall Windows Defender se réinitialise et donc efface les règles de traffics entrants/sortants permettant au 3CX et aux téléphones liés de fonctionner correctement.

\sphinxAtStartPar
De plus, Windows est par défaut plus vulnérable que Linux, de par son architecture et car il est l’OS le plus répandu !

\sphinxAtStartPar
Lorsque l’installation est terminée, on peut remarquer dans le fichier hosts de notre OS Windows que 3CX a rajouté cette ligne :
\begin{quote}

\sphinxAtStartPar
\sphinxcode{\sphinxupquote{127.0.0.1 arthur.3cx.ch}}
\end{quote}

\sphinxAtStartPar
Cette dernière permet, lorsque nous tapons l’URL en question dans notre navigateur, que notre ordinateur pointe vers notre adresse loopback.

\sphinxAtStartPar
Attention, cela se produit seulement si … config préalable dns non

\noindent\sphinxincludegraphics{{3cx-hosts}.png}


\bigskip\hrule\bigskip



\subsection{Interface}
\label{\detokenize{Documentation-M362:interface}}

\subsubsection{Web interface (admin)}
\label{\detokenize{Documentation-M362:web-interface-admin}}
\sphinxAtStartPar
Après avoir terminé la configuration du 3CX, vous pourrez accéder à l’URL correspondante à l’installation de votre 3CX (\sphinxstyleemphasis{ici arthur.3cx.ch:5001}), et ainsi vous logger avec les identifiants administrateur précédemment choisis.

\noindent\sphinxincludegraphics{{3cx-login}.png}

\noindent\sphinxincludegraphics{{dashboard}.png}

\sphinxAtStartPar
Après s’être identifiés, nous débarquons sur l’interface admin.

\sphinxAtStartPar
Pour avoir une ligne entrante et sortante opérationnelle, il est nécessaire de configurer un trunk SIP.
3CX prend en charge plusieurs opérateurs en Suisse, notamment sipcall…

\noindent\sphinxincludegraphics{{sip-trunk}.png}

\sphinxAtStartPar
Ci\sphinxhyphen{}dessus, nos 2 trunks sont déjà configurés. Nous pouvons cependant plonger dans leur configuration afin de comprendre les paramètres incontournables.

\noindent\sphinxincludegraphics{{telco1a}.png}

\noindent\sphinxincludegraphics{{telco1b}.png}


\subsubsection{Web Interface (client)}
\label{\detokenize{Documentation-M362:web-interface-client}}
\sphinxAtStartPar
Il est possible d’accéder à l’interface webclient et ainsi d’avoir des fonctionnalités UCC proposées par 3CX :

\sphinxAtStartPar
Cela inclut :
\begin{itemize}
\item {} 
\sphinxAtStartPar
Chats

\item {} 
\sphinxAtStartPar
Chats de groupe

\item {} 
\sphinxAtStartPar
Meetings (avec caméra, micro, partage d’écran/app…)

\item {} 
\sphinxAtStartPar
Historique des appels

\item {} 
\sphinxAtStartPar
Cahier de contacts

\item {} 
\sphinxAtStartPar
Boîte de messagerie vocale

\end{itemize}

\sphinxAtStartPar
Tout est accessible depuis le menu latérale de gauche :

\noindent\sphinxincludegraphics{{webclient}.png}

\sphinxAtStartPar
Chats :

\sphinxAtStartPar
L’interface des chats est assez rudimentaire mais efficace.
Elle permet de partager des fichiers, faire des listes à puces…

\noindent\sphinxincludegraphics{{webclient-chat}.png}

\sphinxAtStartPar
Chats de groupe :

\sphinxAtStartPar


\sphinxAtStartPar
3CX permet notamment de faire des conférences en ligne, grâce à une interface intuitive et pratique.
Pour pouvoir profiter pleinement de toutes ces fonctionnalités, il est nécessaire d’accorder l’accès au micro et webcam à votre navigateur web.

\sphinxAtStartPar
Durant ces conférences, il est possible de partager son écran et de donner la main à un des collaborateurs présents dans la réunion.
Partager des fichiers et écrire dans un chat dédié est aussi possible !

\noindent\sphinxincludegraphics{{webclient-meeting}.png}

\sphinxAtStartPar
Historique des appels :

\noindent\sphinxincludegraphics{{callhistory}.png}

\sphinxAtStartPar
Cahier de contacts :

\sphinxAtStartPar
Un cahier des contacts existe, donnant la possibilité d’enregistrer des fiches contacts.
Pour aller plus loin, une intégration LDAP est même possible pour télécharger l’annuaire depuis un serveur LDAP. (disponible pour la licence 3CX Pro)

\noindent\sphinxincludegraphics{{phonebook}.png}

\sphinxAtStartPar
Boîte de messagerie vocale :


\subsection{Généralités Réseau}
\label{\detokenize{Documentation-M362:generalites-reseau}}

\bigskip\hrule\bigskip



\subsection{Exigences réseau}
\label{\detokenize{Documentation-M362:exigences-reseau}}
\sphinxAtStartPar
Ce chapitre se base sur le cours 07\sphinxhyphen{}Exigences Réseau du cockpitprofessionnel.ch

\sphinxAtStartPar
\sphinxstylestrong{Latence}

\sphinxAtStartPar
La durée d’exécution des paquets vocaux est un critère essentiel pour la qualité vocale. On s’intéresse ici au délai total entre la parole de l’émetteur et l’écoute du récepteur (délai de bout en bout).

\noindent\sphinxincludegraphics{{latence}.png}

\sphinxAtStartPar


\sphinxAtStartPar
\sphinxstylestrong{Gigue (Jitter)}

\sphinxAtStartPar
Il désigne la différence de délai de transmission de bout en bout entre différents paquets d’un même flux de paquets lors d’une transmission d’un système à l’autre.
Il s’agit en réalité d’une variation de lantence.

\noindent\sphinxincludegraphics{{jitter}.png}

\sphinxAtStartPar


\sphinxAtStartPar
\sphinxstylestrong{Perte de paquets}

\sphinxAtStartPar
Un paquet vocal contient seulement 20 à 30 ms de paroles, ce qui correspond environ à une syllabe. Un codec doit pouvoir compenser jusqu’à 5\% de perte de données, ce qui n’est pas entendu lors d’une conversation téléphonique.

\noindent\sphinxincludegraphics{{pertedepaquets}.png}


\subsection{Fonctions de réseau}
\label{\detokenize{Documentation-M362:fonctions-de-reseau}}

\subsubsection{PoE (Power over Ethernet)}
\label{\detokenize{Documentation-M362:poe-power-over-ethernet}}
\sphinxAtStartPar
La norme IEEE 802.3af, aussi appelée PoE, permet, initialement, de faire passer une alimentation en courant continu d’une puissance de max. 15,4W avec une tension d’environ 48V, en plus des données avec un débit de 100Mbit/s à 1Gbit/s.
Aujourd’hui la norme initiale a évolué (avec le PoE+, et PoE++), permettant de faire passer plus de courant, et donc d’alimenter des appareils de plus en plus gourmands en énergie !

\sphinxAtStartPar
Tableau des normes PoE à voir ci\sphinxhyphen{}dessous :

\noindent\sphinxincludegraphics{{normes-poe}.png}


\bigskip\hrule\bigskip



\subsection{Codecs}
\label{\detokenize{Documentation-M362:codecs}}

\subsubsection{G711}
\label{\detokenize{Documentation-M362:g711}}
\sphinxAtStartPar
Les caractéristiques du codec G.711 sont les suivantes :
\begin{itemize}
\item {} 
\sphinxAtStartPar
Bande de fréquences : 300\sphinxhyphen{}3400Hz

\item {} 
\sphinxAtStartPar
Fréquence d’achantillonnage de 8 khz

\item {} 
\sphinxAtStartPar
Débit fixe de 64 kbits/s (échantillons de 8 bits x 8 kHz)

\item {} 
\sphinxAtStartPar
Délai de compression de 0,125 ms (sans aucun délai d’anticipation)

\end{itemize}

\sphinxAtStartPar
MOS :
\begin{itemize}
\item {} 
\sphinxAtStartPar
Mesure de qualité en conditions idéales : MOS a revoir

\item {} 
\sphinxAtStartPar
Mesure de qualité en condition dégradées : MOS a revoir

\end{itemize}

\sphinxAtStartPar
Pour tout appel passant par IP, une initiation de communications est procédé par le protocole SIP.
Ce dernier pourrait être comparable au fonctionnement du TCP, mais à la couche 7 du modèle OSI.

\sphinxAtStartPar
Capture wireshark d’une conversation en G711 (flux RTP):

\noindent\sphinxincludegraphics{{rtp-conf-payload-G711}.png}

\sphinxAtStartPar
Comme escompté, nous remarquons que la discussion transite du téléphone 192.168.1.122 en passant par le serveur 3CX 192.168.1.120 .

\sphinxAtStartPar
La première chose qui est importante à souligner, est que les paquets utilisent le protocole de transport UDP (couche OSI 4) pour naviguer à travers le réseau, réduisant donc la latence potentielle de la conversation.

\sphinxAtStartPar
Étant donné que le trafic est d’interne à interne, il n’est par défaut pas chiffré, laissant le payload contenu dans le RTP visible en clair.
Il est donc tout à fait possible à partir d’un fichier d’un logiciel tel que Wireshark, d’écouter une conversation à partir de la conversation RTP !

\noindent\sphinxincludegraphics{{i2i-call-RTP-voice-recording}.png}


\subsubsection{G722}
\label{\detokenize{Documentation-M362:g722}}
\sphinxAtStartPar
Les caractéristiques du codec G.722 sont les suivantes :
\begin{itemize}
\item {} 
\sphinxAtStartPar
Bande de fréquences : 50\sphinxhyphen{}7000Hz

\item {} 
\sphinxAtStartPar
Fréquence d’échantillonnage : 16 kHz

\item {} 
\sphinxAtStartPar
Débit fixe : 64 kbps

\item {} 
\sphinxAtStartPar
Délai de compression : Non spécifié

\end{itemize}

\sphinxAtStartPar
MOS :
\begin{itemize}
\item {} 
\sphinxAtStartPar
Mesure de qualité en conditions idéales : MOS (Mean Opinion Score) similaire pour G.722 et G.711

\item {} 
\sphinxAtStartPar
Mesure de qualité en conditions dégradées : MOS (Mean Opinion Score) similaire pour G.722 et G.711

\end{itemize}

\sphinxAtStartPar
Voici un graphique comparatif pour les bandes de fréquence du G711 et du G722 :

\noindent\sphinxincludegraphics{{g711-g722-frequency-response}.jpg}

\sphinxAtStartPar


\noindent\sphinxincludegraphics{{rtp-conf-payload-G722}.png}

\sphinxAtStartPar



\subsubsection{G729}
\label{\detokenize{Documentation-M362:g729}}
\noindent\sphinxincludegraphics{{rtp-conf-payload-G729}.png}

\sphinxAtStartPar
Les caractéristiques du codec G.722 sont les suivantes :
\begin{itemize}
\item {} 
\sphinxAtStartPar
Bande de fréquences : 50\sphinxhyphen{}7000Hz

\item {} 
\sphinxAtStartPar
Fréquence d’échantillonnage : 16 kHz

\item {} 
\sphinxAtStartPar
Débit fixe : 64 kbps

\item {} 
\sphinxAtStartPar
Délai de compression : Non spécifié

\end{itemize}

\sphinxAtStartPar
MOS :
\begin{itemize}
\item {} 
\sphinxAtStartPar
Mesure de qualité en conditions idéales : MOS (Mean Opinion Score) similaire pour G.722 et G.711

\item {} 
\sphinxAtStartPar
Mesure de qualité en conditions dégradées : MOS (Mean Opinion Score) similaire pour G.722 et G.711

\end{itemize}

\sphinxAtStartPar
Parler de la MOS pour la qualité audio


\bigskip\hrule\bigskip



\bigskip\hrule\bigskip



\section{Exercices}
\label{\detokenize{Documentation-M362:exercices}}

\subsection{Exercice 1}
\label{\detokenize{Documentation-M362:exercice-1}}

\subsubsection{Demande}
\label{\detokenize{Documentation-M362:demande}}
\sphinxAtStartPar
\sphinxstylestrong{Exercice 1: Création d’un numéro d’assistance}

\sphinxAtStartPar
L’accessibilité téléphonique du service clientèle de Cardinal Bier Import AG doit être améliorée. À l’heure actuelle, le numéro principal n’est desservi que par une seule personne. Récemment, une application de Customer Releationship Management a été installée. Désormais, les commandes, réclamations ou autres demandes des clients sont enregistrées électroniquement. Une équipe de 4 collaboratrices a été formée. La répartition des appels au sein de cette équipe doit être définie. Créez une solution de téléphonie pour le service clientèle de Cardinal Bier Import AG. Vous disposez d’une instance vPBX de Peoplefone ou d’autres installations. Lisez les exigences de l’entreprise et établissez une configuration.

\sphinxAtStartPar
\sphinxstylestrong{Besoins en téléphonie du service clientèle}

\sphinxAtStartPar

\begin{itemize}
\item {} 
\sphinxAtStartPar
Horaires d’ouverture du lundi au vendredi de 8h00 à 18h00 et le samedi de 8h00 à 17h00

\item {} 
\sphinxAtStartPar
Saisie de tous les jours fériés catholiques légaux pour le site de Fribourg, pour les 12 prochains mois.

\item {} 
\sphinxAtStartPar
IVR pour allemand, français et anglais en amont

\end{itemize}

\sphinxAtStartPar
Formez des groupes pertinents. Les appels doivent être répartis de manière séquentielle au sein du groupe. Il doit y avoir passage d’un groupe à un autre, si personne ne répond ou si la ligne est occupée. L’appel passera sur messagerie et signalera qu’aucun collaborateur n’est libre, seulement aucune personne ne répond. Les équipes parlant les langues officielles du canton reçoivent un numéro d’appel externe et les collaboratrices peuvent passer des appels externes sur lle téléphone IP avec ce numéro ou avec le numéro principal.

\sphinxAtStartPar
Les textes de message suivants peuvent être repris dans le fichier ZIP ou vous pouvez en enregistrer vous\sphinxhyphen{}même:
\begin{itemize}
\item {} 
\sphinxAtStartPar
HPN\_AB\_FeiertagFerien.wav

\item {} 
\sphinxAtStartPar
HPN\_AB\_keinMitarbeiterFrei.wav

\item {} 
\sphinxAtStartPar
HPN\_AB\_Oefffnungszeiten.wav

\item {} 
\sphinxAtStartPar
IVR\_Ansage.wav

\end{itemize}

\sphinxAtStartPar
Fichiers WAV
Le texte parlé des fichiers WAV ne doit pas correspondre à 100\% à la problématique de cet exercice.
Les utilisateurs suivants doivent être enregistrés:
\begin{itemize}
\item {} 
\sphinxAtStartPar
Meier Anna, parle allemand, français

\item {} 
\sphinxAtStartPar
Müller Janine, parle allemand, anglais

\item {} 
\sphinxAtStartPar
Angeloz Marie, parle français

\item {} 
\sphinxAtStartPar
Ducrest Sophie, parle français, anglais

\end{itemize}

\sphinxAtStartPar
Mission par groupe de 2 ou 4:
\begin{itemize}
\item {} 
\sphinxAtStartPar
Tracez le Call Flow pour le numéro principal (modèles disponibles dans le chapitre 10 du module 361)

\item {} 
\sphinxAtStartPar
Configurez l’installation en fonction des exigences

\end{itemize}

\sphinxAtStartPar
Testez l’installation et consignez les tests dans un protocole


\subsubsection{Workflow :}
\label{\detokenize{Documentation-M362:workflow}}
\sphinxAtStartPar
La chose la plus importante à faire dans un exercice tel quel, est de dessiner un schéma de principe très simple, à la main de préférence.

\sphinxAtStartPar
Cela permet de visualiser au mieux la demande et de pouvoir poser des questions au client si les indications ne sont pas claires !

\noindent\sphinxincludegraphics{{schema-ex1}.png}

\sphinxAtStartPar
La demande est désormais plus compréhensible, nous allons donc maintenant procéder à la configuration de notre PBX virtuel !

\sphinxAtStartPar
Commencons par les utilisateurs :

\sphinxAtStartPar


\noindent\sphinxincludegraphics{{users1}.png}

\sphinxAtStartPar
Configuration Janine :

\noindent\sphinxincludegraphics{{janine}.png}

\sphinxAtStartPar
Les champs obligatoires à remplir lors de la création de l’utilisateur sont les suivants :
\begin{itemize}
\item {} 
\sphinxAtStartPar
Extension

\item {} 
\sphinxAtStartPar
Prénom

\item {} 
\sphinxAtStartPar
Nom

\item {} 
\sphinxAtStartPar
Adresse Mail

\end{itemize}


\bigskip\hrule\bigskip



\subsection{Exercice 2}
\label{\detokenize{Documentation-M362:exercice-2}}

\subsubsection{1. NAT / PAT avec installation app natel externe}
\label{\detokenize{Documentation-M362:nat-pat-avec-installation-app-natel-externe}}
\sphinxAtStartPar
Workflow de l’exercice :

\sphinxAtStartPar
Dépannage 3CX

\noindent\sphinxincludegraphics{{depannage-3cx}.png}

\sphinxAtStartPar
Vous avez la possibilité à travers ce menu de définir si oui ou non le serveur 3CX agit en tant qu’intermédiaire pour les appels.
Ici, cela nous sera utile afin de nous simplifier la tâche, au lieu de configurer un port de mirroring sur le switch par exmple.

\sphinxAtStartPar
La prochaine étape sera de créer la règle NAT/PAT dans le routeur / firewall du réseau (ici Centro Business 2.0 Swisscom)
Nous accédons donc à la web interface administrateur de ce dernier (Réseau\textgreater{}Port Forwarding\textgreater{} Create new rule)
\begin{itemize}
\item {} 
\sphinxAtStartPar
Port TCP 5001 (HTTPS)

\item {} 
\sphinxAtStartPar
Port TCP/UDP 5090 (Tunnel 3CX)

\end{itemize}

\noindent\sphinxincludegraphics{{natpat-swisscom-ex2}.png}

\sphinxAtStartPar
A la suite de cette configuration, nous pouvons télécharger l’application 3CX sur notre téléphone.

\sphinxAtStartPar
{\color{red}\bfseries{}|:warning:|} DISCLAIMER {\color{red}\bfseries{}|:warning:|}

\sphinxAtStartPar
Sur Android, l’application \sphinxstylestrong{nécessite} le GSF afin d’afficher les notifications d’appels entrants.
Dans le cas contraire, vous ne pourrez pas répondre aux appels, mais serez en mesure d’en passer (appels sortants).

\sphinxAtStartPar
Précision faite, il est temps d’installer l’application sur notre appareil !

\sphinxAtStartPar
Rendez\sphinxhyphen{}vous dans votre gestionnaire de paquets / applications préféré \textgreater{} Tapez 3CX dans la barre de recherche \textgreater{} Installez l’application

\noindent\sphinxincludegraphics[width=240\sphinxpxdimen]{{dwn-3cx}.jpg}

\sphinxAtStartPar
Ensuite, lisez et acceptez les conditions d’utilisation de l’app.

\noindent\sphinxincludegraphics[width=240\sphinxpxdimen]{{open3cx-1}.png}

\sphinxAtStartPar
Pour finir, scannez le QR code que vous trouvez dans la configuration de votre utilisateur 3CX.

\noindent\sphinxincludegraphics[width=240\sphinxpxdimen]{{open3cx-2}.png}

\sphinxAtStartPar
Vous êtes désormais connecté à votre compte, vous permettant donc de passer des appels et d’envoyer des messages dans le service de chat 3CX.


\subsubsection{2. 1 App + 1 Webclient en interne avec Wireshark}
\label{\detokenize{Documentation-M362:app-1-webclient-en-interne-avec-wireshark}}

\subsubsection{3. 2 Téléphones SIP avec Wireshark (comparaison G711/G722/G729 )}
\label{\detokenize{Documentation-M362:telephones-sip-avec-wireshark-comparaison-g711-g722-g729}}
\sphinxstepscope


\chapter{Module 144 : Réseaux sans\sphinxhyphen{}fil (en travaux)}
\label{\detokenize{Documentation-M144:module-144-reseaux-sans-fil-en-travaux}}\label{\detokenize{Documentation-M144::doc}}

\section{Introduction}
\label{\detokenize{Documentation-M144:introduction}}
\sphinxAtStartPar
La première leçon du module était une introduction à ce dernier. Elle nous a permis de tester nos connaissances sur diverses technologies sans\sphinxhyphen{}fil afin de nous mettre en bouche les prochaines leçons qui suivront.

\sphinxAtStartPar
Durant la première année, lors du module M145, nous avions déjà eu la chance d’étudier les différentes caractéristiques du Wi\sphinxhyphen{}Fi, notamment :
\begin{itemize}
\item {} 
\sphinxAtStartPar
Ses topologies

\item {} 
\sphinxAtStartPar
Ses normes

\item {} 
\sphinxAtStartPar
Son évolution dans le temps

\item {} 
\sphinxAtStartPar
Les mécanismes d’association et de transfert des données

\item {} 
\sphinxAtStartPar
Sa sécurité

\item {} 
\sphinxAtStartPar
Ses avantages

\item {} 
\sphinxAtStartPar
Ses inconvénients

\end{itemize}

\sphinxAtStartPar
Le Wi\sphinxhyphen{}Fi est donc une technologie qui ne nous est pas inconnue, bien que nous l’ayons plus ou moins survolée.

\sphinxAtStartPar
C’est ici que le module M144 intervient ; pour rentrer en profondeur dans cette technologie qui nous entoure quotidiennement.

\sphinxAtStartPar
Ici, quelques QCM disponibles sur eitswiss.cockpitprofessionnel.ch concernant les technologies sans\sphinxhyphen{}fil

\noindent{\hspace*{\fill}\sphinxincludegraphics{{qcm-1}.png}\hspace*{\fill}}


\section{Semaine 2}
\label{\detokenize{Documentation-M144:semaine-2}}

\subsection{Introduction}
\label{\detokenize{Documentation-M144:id1}}
\sphinxAtStartPar
Voici la liste des sujets abordés durant la semaine 2 :
\begin{itemize}
\item {} 
\sphinxAtStartPar
La fréquence

\item {} 
\sphinxAtStartPar
La bande passante

\item {} 
\sphinxAtStartPar
Le débit binaire

\item {} 
\sphinxAtStartPar
Les différents types de modulations ( 5 )

\item {} 
\sphinxAtStartPar
Étalement de spectre ( 6 )

\item {} 
\sphinxAtStartPar
Multiplexage ( 7 )
\begin{quote}

\sphinxAtStartPar
OFDM
FHSS
DSSS
Les normes 802.11
\end{quote}

\end{itemize}


\subsection{Fréquence}
\label{\detokenize{Documentation-M144:frequence}}
\sphinxAtStartPar
Qu’est\sphinxhyphen{}ce que la fréquence (Hz) ?
\begin{itemize}
\item {} 
\sphinxAtStartPar
La fréquence (Hz) est le nombre de périodes (oscillations) qui se répète en une période de temps t.

\end{itemize}

\sphinxAtStartPar
Comment la calcule\sphinxhyphen{}t\sphinxhyphen{}on ?

\sphinxAtStartPar
f=  1/(t (en s))

\sphinxAtStartPar
Un exemple concret et graphique :

\noindent{\hspace*{\fill}\sphinxincludegraphics{{sinus-1}.png}\hspace*{\fill}}


\subsection{Bande passante}
\label{\detokenize{Documentation-M144:bande-passante}}
\sphinxAtStartPar
La bande passante représente la bande de fréquence dans laquelle peuvent passer les données.

\sphinxAtStartPar
Nous pouvons faire une analogie avec une autoroute qui a tant de voies pour faire passer tant de voitures.

\sphinxAtStartPar
Attention : La bande passante est à ne pas confondre avec le débit, bien que les deux aient un lien :
\begin{itemize}
\item {} 
\sphinxAtStartPar
Une bande passante élevée ne garantit pas nécessairement un débit élevé, tandis qu’un débit élevé est toujours le résultat d’une bande passante élevée.

\end{itemize}


\subsection{Débit binaire}
\label{\detokenize{Documentation-M144:debit-binaire}}
\sphinxAtStartPar
Le débit binaire, souvent simplement appelé « débit, » est la mesure de la quantité d’informations numériques (bits) transmises ou traitées par unité de temps, généralement en bits par seconde (bps).

\noindent{\hspace*{\fill}\sphinxincludegraphics{{speedtest}.jpg}\hspace*{\fill}}


\subsection{Modulation}
\label{\detokenize{Documentation-M144:modulation}}
\sphinxAtStartPar
Qu’est\sphinxhyphen{}ce que la modulation ?

\sphinxAtStartPar
La modulation est le processus de modification d’une onde porteuse afin qu’elle puisse porter des informations (data, voix…), sur un canal de communication.

\sphinxAtStartPar
Y a\sphinxhyphen{}t\sphinxhyphen{}il plusieurs types de modulation existants dans le monde des télécommunications ?

\sphinxAtStartPar
Oui, les voici :
\begin{itemize}
\item {} 
\sphinxAtStartPar
Modulation d’amplitude (AM)
\begin{quote}

\sphinxAtStartPar
La modulation d’amplitude consiste à moduler l’amplitude d’un signal porteur.
Exemple concret ci\sphinxhyphen{}dessous :

\noindent{\hspace*{\fill}\sphinxincludegraphics{{am}.png}\hspace*{\fill}}
\end{quote}

\item {} 
\sphinxAtStartPar
Modulation de fréquence (FM)

\item {} 
\sphinxAtStartPar
Modulation de phase (PM)

\end{itemize}


\subsection{Normes 802.11}
\label{\detokenize{Documentation-M144:normes-802-11}}
\sphinxAtStartPar
La norme 802.11 est une série de normes qui spécifient les protocoles de communication sans fil pour les réseaux locaux (Wi\sphinxhyphen{}Fi). Ces normes ont été développées par l’IEEE, un organisme de normalisation international. La famille de normes 802.11 définit les spécifications techniques pour les réseaux sans fil, notamment les fréquences, les débits de données, les protocoles de sécurité, etc.

\sphinxAtStartPar
Les normes 802.11 ont évolué au fil du temps pour s’améliorer et permettre :
\sphinxhyphen{} Des débits plus élevés
\sphinxhyphen{} Plus de fiabilité
\sphinxhyphen{} Plus de sécurité
\sphinxhyphen{} Plus de bande de fréquences

\sphinxAtStartPar
Certaines des versions les plus couramment connues de la norme 802.11 incluent :
\sphinxhyphen{} 802.11a
\sphinxhyphen{} 802.11b
\sphinxhyphen{} 802.11g
\sphinxhyphen{} 802.11n
\sphinxhyphen{} 802.11ac
\sphinxhyphen{} 802.11ax
\sphinxhyphen{} 802.11ay


\section{Semaine 3}
\label{\detokenize{Documentation-M144:semaine-3}}

\subsection{Introduction}
\label{\detokenize{Documentation-M144:id2}}
\sphinxAtStartPar
Voici les différents sujets abordés lors de la 3ème semaine de cours sur le module M144 :
\begin{itemize}
\item {} 
\sphinxAtStartPar
Tableau comparatif des technologies sans fil (suite)

\item {} 
\sphinxAtStartPar
Le roaming

\item {} 
\sphinxAtStartPar
Organismes de normalisation

\item {} 
\sphinxAtStartPar
La trame 802.11

\item {} 
\sphinxAtStartPar
Les topologies

\item {} 
\sphinxAtStartPar
Étalement de spectre ( 6 )

\item {} 
\sphinxAtStartPar
Multiplexage ( 7 )

\end{itemize}


\subsection{Le roaming}
\label{\detokenize{Documentation-M144:le-roaming}}
\sphinxAtStartPar
Il est possible d’exploiter deux points d’accès (AP1 et AP2) avec des zones de couverture différentes mais le même SSID et le même réseau W\sphinxhyphen{}LAN. Ces deux AP sont câblés avec le même switch. Si un terminal actuellement connecté au point d’accès AP1 via le SSID « Edu\_WLAN1 » est déplacé en direction du point d’accès AP2, le signal du point d’accès AP1 s’affaiblit soudainement et celui du point d’accès AP2 s’intensifie. Le terminal se connecte désormais presque de manière ininterrompue à AP2. Ce procédé est appelé roaming. L’utilisateur n’est au courant de rien. Idéalement, AP1 et AP2 (et éventuellement d’autres AP) ont une plage qui se chevauche. La répartition roaming convient aux zones de couverture plus grandes, telles que dans des moyennes et grandes entreprises ou dans des écoles.


\subsection{Trame 802.11}
\label{\detokenize{Documentation-M144:trame-802-11}}
\sphinxAtStartPar
Afin de pouvoir comprendre de quoi est composé une trame 802.11, il est intéressant de se pencher sur la trame Ethernet II (802.3), ces dernières ayant, non seulement, beaucoup de similitudes, mais aussi, plusieurs différences conséquentes telles que :
\begin{itemize}
\item {} 
\sphinxAtStartPar
La différence de taille :
\sphinxhyphen{} 802.3 : 1542 octets
\sphinxhyphen{} 802.11 : 2312 octets

\item {} 
\sphinxAtStartPar
La méthode d’accès au média :
\sphinxhyphen{} 802.3 : CSMA\sphinxhyphen{}CD
\sphinxhyphen{} 802.11 : CSMA\sphinxhyphen{}CA

\end{itemize}


\subsection{Topologies \& Environnement}
\label{\detokenize{Documentation-M144:topologies-environnement}}
\sphinxAtStartPar
Différentes topologies existent pour les réseaux sans\sphinxhyphen{}fil, ces dernières permettant une flexibilité dans l’adaptation des besoins des clients.

\sphinxAtStartPar
IBSS :

\sphinxAtStartPar
BSS :

\sphinxAtStartPar
ESS :

\sphinxAtStartPar
SOHO :

\sphinxAtStartPar
Il s’agit ici d’un routeur W\sphinxhyphen{}LAN usuel. C’est un appareil très performant, qui intègre certains niveaux de fonction et qui se trouve dans pratiquement tous les foyers et/ou petits bureau (small office). Ce routeur W\sphinxhyphen{}LAN intègre un switch, un modem Internet (DSL, câble, 4G, 5G), un serveur DHCP, un pare\sphinxhyphen{}feu et un point d’accès pour la connexion sans fil. L’un des représentants les plus populaires de cette catégorie est la « Fritzbox ». Le routeur W\sphinxhyphen{}LAN est un ESS en lui\sphinxhyphen{}même.

\sphinxAtStartPar
Cependant, il est important de notifier que l’usage de répéteur afin d’augmenter la couverture de votre W\sphinxhyphen{}LAN est possible.

\sphinxAtStartPar
Mais attention car l’usage d’un seul répéteur permet de garder un débit élevé car il dirige le signal vers un autre canal, mais tout autre répéteur ajouté divisera le débit par 2.

\sphinxAtStartPar
C’est donc une solution de dernier recours si rien d’autre est possible.

\sphinxAtStartPar
Nous allons maintenant nous intéressons à l’environnement entourant notre AP et pouvant éventuellement causer des perturbations ou des atténuations sur nos signaux.

\sphinxAtStartPar
Avant toute chose, il est important de comparer les fréquences utilisées pour la technologie 802.11.


\subsection{Mandat pratique 30.3.5}
\label{\detokenize{Documentation-M144:mandat-pratique-30-3-5}}
\sphinxAtStartPar
Quelques questions du cockpit :


\subsection{Mandat pratique IBSS}
\label{\detokenize{Documentation-M144:mandat-pratique-ibss}}
\sphinxAtStartPar
Afin de comprendre dans quels domaines d’applications nous pouvons utiliser la topologie IBSS, il nous a été demandé de réaliser un partage de fichier soit :
\begin{itemize}
\item {} 
\sphinxAtStartPar
Par AirDrop (technologie Apple)

\item {} 
\sphinxAtStartPar
Par Wifi Direct (disponible sur les smartphones sous Android)

\end{itemize}

\sphinxAtStartPar
Ayant un iPhone, j’ai décidé de compléter le mandat en utilisant AirDrop :


\chapter{Indices and tables}
\label{\detokenize{index:indices-and-tables}}\begin{itemize}
\item {} 
\sphinxAtStartPar
\DUrole{xref,std,std-ref}{genindex}

\item {} 
\sphinxAtStartPar
\DUrole{xref,std,std-ref}{modindex}

\item {} 
\sphinxAtStartPar
\DUrole{xref,std,std-ref}{search}

\end{itemize}



\renewcommand{\indexname}{Index}
\printindex
\end{document}